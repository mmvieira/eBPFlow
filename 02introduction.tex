\section{Introduction}
\label{sec:intro}

Software-Defined Networking (SDN) is a paradigm for the development of research in computer networks that has gained the attention of the scientific community and industry in the area. SDN is a paradigm that separates the control plane from the data plane and allows the administrator to program the devices~\cite{ProgrammableNetworks2015}. In SDN, the control plane configures the routing rules of the network with a logically centralized entity called controller, while the data plane forwards the packets according to the actions defined by this controller. Due to the structure that SDN provides, research areas such as traffic engineering, quality of service (QoS), and virtualization have evolved rapidly~\cite{stubbe2017p4}.


 
% The OpenFlow~\cite{McKeown:2008:OpenFlow} standard is an example of SDN, which has seen significant growth since its first release in 2008 until the release of the current version (1.5)~\cite{ChristianSurveySDN2015}. The first version of OpenFlow had a matching table of ten fields and evolved into multiple tables with 44 different fields~\cite{ChristianSurveySDN2015}. However, the number of fields supported by OpenFlow is constantly being updated to support new fields and protocols, such as the IPv6 protocol. Unfortunately, OpenFlow has a protocol-dependent data plane with parsing, matching, and fixed actions, making it difficult to support new fields and protocols~\cite{Jouet:2017:BPFabric}.

\begin{figure}[!htp]
\centering
\includegraphics[width=.6\linewidth]{figures/"eBPFlowHierarchicalLanguages".pdf}
\caption{Expressiveness}
\label{fig:Language}
\end{figure}

Figure~\ref{fig:Language} presents the expressiveness of current designs. The first version of OpenFlow~\cite{McKeown:2008:OpenFlow} standard had a matching table of ten fields and evolved into multiple tables with 44 different fields in the release of version (1.5)~\cite{ChristianSurveySDN2015}. Unfortunately, OpenFlow has a protocol-dependent data plane with parsing, matching, and fixed actions, making it difficult to support new fields and 
protocols~\cite{Jouet:2017:BPFabric}.
P4~\cite{Bosshart:2014:P4} provides a programmable parser, enabling to match new fields. P4 follows the match-action abstraction model where network functions are described as a pipeline of a programmable parser and stages of match and action components. P4-16~\cite{P4-16} provides a primitive to store state using small registers but this is too restrict for large range of stateful functions. OpenState~\cite{Bianchi:2014:OPP:2602204.2602211} and FAST (Flow-level State Transitions)~\cite{Moshref:2014:FST:2620728.2620729} allow to program by explicitly defining flow state using Finite State Machine (FSM). But FSMs need to define all possible states, which does not scale. Extended Finite State Machines (EFSM) enhances the FSM model by allowing variables to describe state and trigger transactions. FlowBaze~\cite{FlowBlaze2019} allows stateful packet processing in hardware by programming using EFSM. \marcos{improve description of eBPF-v and eBPF.} eBPF-verifier is the set of eBPF programs that passes the eBPF-verifier. The verifier does not allow backward jumps, thus, loops have to be unrolled. eBPF and x86 instructions provide Turing-Complete execution model.

A protocol-independent packet processor is a device that does not have the protocol specification a priori. It does not know how to process a specific protocol until the programmer tells how to do it. The packet processor works as a substrate to process packets but it is not tied to a given protocol. Protocol-independent packet processor enables including recent protocols such as NVGRE~\cite{rfc7637}, VxLAN~\cite{mahalingam2013}, STT~\cite{davie2014stt}, PBB~\cite{kishjac-bmwg-evpntest-08}, OTV~\cite{hasmit-otv-04}, GENEVE~\cite{ietf-nvo3-geneve-05}, and NSH~\cite{rfc8300} without hardware modification.
% It also permit to customize what protocols to process. For instance, Amazon Virtual Private Cloud (VPC) was reported to only performs unicast IPv4 forwarding~\cite{Amazon2013Pepelnjak}. Thus, data center providers can benefit from protocol-independent switches by customizing only the necessary protocols they need for packet processing.

% PISCES~\cite{Shahbaz:2016:Pisces} is a software protocol-independent switch that overcomes some of the OpenFlow limitations by permitting relational comparison with >, <, and logical operator and. The P4-to-OVS compiler in PISCES outputs C source code that replaces
% the parsing, match, and action code in OVS~\cite{Pfaff:2015:OpenVSwitch}. But, to change the functionality of the switch, PISCES requires to recompile the software switch and reinstall it.

Moreover, hardware devices give the impression of being inflexible since their components are hard-coded and hard-placed. We address the following question: Is it possible to design a programmable protocol-independent packet processor in hardware that enables to modify the parsing, matching, and actions at runtime? 

 \begin{figure}[!htp]
 \centering
\includegraphics[width=.8\linewidth]{figures/"eBPFlow Design Space".pdf}
 \caption{Design Space}
 \label{fig:Comparison}
 \end{figure}

Figure~\ref{fig:Comparison} compares the design space.
\marcos{Need to finish descrption of figure here}

%\textcolor{blue}{Add eBPF explanation}

This work proposes eBPFlow, a packet processor implemented in hardware, that allows the utilization of new dynamically defined fields and protocols, without the need to recompile or restart the device when the user changes at run time how the flows should be processed. 
eBPFlow leverages eBPF (enhanced Berkeley Packet Filter), a bytecode machine and protocol independent instruction set. % to define parsing, matching, and functionalities of the data plane.
The device was prototyped on the NetFPGA SUME 40 Gb/s~\cite{SUME2014} platform. The tests were performed in a real environment. eBPFlow allows to parse, match, and make actions in the data plane at runtime with zero downtime.

%\textcolor{blue}{vantagem de zero downtime}
Zero downtime is important to provide services without interruptions.
Google~\cite{Jain:2013:BEG:2486001.2486019} reported a substantial outage when doing maintenance. In  Amazon's platform Dynamo~\cite{DeCandia:2007:DAH:1294261.1294281},  services have stringent requirements to meet the customer's Service Level Agreements (SLA).
An outage can mean lost of money for not providing the service or even real lost sales on e-commerce sites.



The main contributions of this work are: 
(i) logic design and hardware implementation of eBPFlow, built on top of the NetFPGA SUME that can forward packets at 40 Gb/s line rate.
(ii) \system enables the programming of stateful and stateless network functions and the use of dynamically defined new fields and protocols, with zero downtime when the user changes, at run time, how the flows should be processed.
(iii) a comprehensive study of \system expressiveness and capabilities through the implementation of many network function cases.\marcos{describe what are the network functions here}

We emphasize that all these contributions together are not possible in other systems, such as BPFabic~\cite{Jouet:2017:BPFabric}, PISCES~\cite{Shahbaz:2016:Pisces}, P4FPGA~\cite{p4fpga}, RMT~\cite{bosshart2013forwarding}, and dRMT~\cite{chole2017drmt}. 

This paper is organized as follows. In \textsection\ref{sec:architecture}, we present the overall architecture of eBPFlow. In \textsection\ref{sec:design}, we provide the eBPFlow packet processor. In \textsection\ref{sec:implementation}, we describe the implementation details of eBPFlow built on top of the NetFPGA SUME platform. In \textsection\ref{sec:results}, we show the evaluation and results in a realistic environment. 
In \textsection\ref{sec:discussion}, we provide some analysis and discuss some limitations.
In \textsection\ref{sec:relatedWork}, we describe and compare the related work. Finally, in \textsection\ref{sec:conclusion}, we present the conclusion and future work.
