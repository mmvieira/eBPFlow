\section{Conclusion and Future Work}
\label{sec:conclusion}

We presented \system, a Turing-Complete hardware packet processing targeted for high-performance data plane network functions.
A packet processing with an eBPF virtual machine at its core was designed and implemented in hardware. 
This system allows executing parse, matching, and actions dynamically through eBPF instructions.
The system is protocol independent and allows to use new fields, facilitating the adoption of new protocols and services in computer networks.
%EBPF instructions are generated from programs written in C or P4 language. 
The \system allows changing the image of the eBPF program at runtime, allowing to modify how the flows should be processed with zero downtime.

% For future work, it is desirable to place an eBPF processor for each input queue. Next step includes producing eBPFlow Application Specific Integrated Circuit (ASIC) and commercializing it.
% We envision that eBPF instructions will replace OpenFlow standard matching and eBPFlow will be used as smart NIC, switches, and routers.

\system is built on top of the NetFPGA SUME platform.
FPGAs are useful for prototyping and validating the hardware logic design~\cite{cofer2005rapid,sukhwani2017contutto}. 
The logic design for FPGA shares the same initial flow and methodology as Application Specific Integrated Circuit (ASIC) design~\cite{kuon07measuring,ASICDesignFlow2005}. 
We leave eBPFlow over ASIC as future work.
We plan to make the eBPFlow implementation publicly available.
